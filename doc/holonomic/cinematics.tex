% Déclaration du type de document (report, book, paper, etc...)
\documentclass[a4paper]{paper} 
 
% Package pour avoir Latex en français
\usepackage[utf8]{inputenc}
\usepackage[francais]{babel}
 
% Quelques packages utiles
\usepackage{listings} % Pour afficher des listings de programmes
\usepackage{graphicx} % Pour afficher des figures
\usepackage{amsthm}   % Pour créer des théorèmes et des définitions
\usepackage{amsmath}
\usepackage{microtype} % Optical margins FTW

\lstset{language=MATLAB}



% Auteur
\author{
Mathieu Rouvinez \\
\texttt{mathieu.rouvinez@gmx.net}
\and
Antoine Albertelli \\
\texttt{antoine.albertelli@gmail.com}
}
 
% Titre du document
\title{Equations de mouvement pour 3-omniwheels}

% Début du document
\begin{document}

\maketitle
% \tableofcontents
\section{Hypothèses}
\begin{itemize}
    \item Les roues sont situées aux sommets d'un triangle équilatéral avec 120\degre entre elles.
    \item Chaque roue peut avoir un rayon différent.
    \item Chaque roue peut se trouver à une distance différente du centre du triangle équilatéral.
    \item Le référentiel du robot se trouve au centre du triangle équilatéral.
\end{itemize}

\section{Cinématique directe}

\subsection{Translation}

\begin{equation}
    \omega_{i,t} = \frac{V}{r_i} \cdot \cos\left( \theta_R - \theta_V + \beta_i - \frac{\pi}{2} \right)
    \label{eqn:forward-translation}
\end{equation}

\begin{description}
    \item[$\omega_{i,t}$] Vitesse de rotation de la roue $i$ pour la translation de la base holonome.
    \item[$v$] Vitesse de déplacement du robot.
    \item[$\theta_R$] Angle absolu de l'orientation du robot.
    \item[$\theta_V$] Angle absolu de la direction de vitesse du robot.
    \item[$\beta_i$] Angle relatif au robot de l'orientation de la roue $i$.
    \item[$r_i$] Rayon de la roue $i$.
\end{description}
\subsubsection{Exemple en MATLAB}

\begin{lstlisting}
beta1 = 60*pi/180;
beta2 = 180*pi/180;
beta3 = 300*pi/180;
omega1_t = linear_speed_value / r1 * ...
           cos(heading-speed_angle+beta1-pi/2);
omega2_t = linear_speed_value / r2 * ...
           cos(heading-speed_angle+beta2-pi/2);
omega3_t = linear_speed_value / r3 * ...
           cos(heading-speed_angle+beta3-pi/2);
\end{lstlisting}

\subsection{Rotation}
\begin{equation}
    \omega_{i, r} = - \Omega \cdot \frac{D_i}{r_i}
    \label{eqn:forward-rotation}
\end{equation}

\begin{description}
    \item[$\omega_{i,r}$] Vitesse de rotation de la roue $i$ pour la rotation de la base holonome.
    \item[$\Omega$] Vitesse de rotation du robot sur lui même.
    \item[$D_i$] Distance entre le plan de rotation de la roue et le centre du robot.
    \item[$r_i$] Rayon de la roue $i$.
\end{description}
\subsubsection{Exemple en MATLAB}
\begin{lstlisting}
omega1_r = -angular_speed_value * D1 / r1;
omega2_r = -angular_speed_value * D2 / r2;
omega3_r = -angular_speed_value * D3 / r3;
\end{lstlisting}

\subsection{Combinaison}
En combinant \eqref{eqn:forward-translation} et \eqref{eqn:forward-rotation}, on obtient l'équation
pour un mouvement composé d'une translation et d'une rotation.
\begin{equation}
    \omega_i = \omega_{i,t} + \omega_{i,r}
    \label{eqn-forward-complete}
\end{equation}
\section{Cinématique inverse}
\subsection{Rotation}
\begin{equation}
    \theta_{R, t+1} = \theta_{R, t} - \frac{1}{steps} \frac{\sum\limits_{i=1}^3 steps_i \cdot r_i}{\sum\limits_{i=1}^3 r_i}
    \label{eqn:reverse-heading}
\end{equation}
\begin{description}
    \item[$\theta_R$] Orientation absolue du robot.
    \item[$steps$] Nombre de pas pour un tour de roue.
    \item[$steps_i$] Avancement de la roue $i$ en steps codeur.
    \item[$r_i$] Distance du plan de rotation de la roue $i$ au centre du robot.
\end{description}

\subsubsection{Exemple en MATLAB}
\begin{lstlisting}
heading = heading-(steps1*r1+steps2*r2+steps3*r3) ...
          /(D1+D2+D3)/steps_turn;
\end{lstlisting}

\subsection{Translation}
\begin{subequations}
    \begin{equation}
    \Delta_x = \frac{2}{3} \frac{1}{steps} \sum\limits_{i=1}^3 \cos \beta_i \cdot steps_i \cdot r_i 
    \end{equation}
    \begin{equation}
    \Delta_y = \frac{2}{3} \frac{1}{steps} \sum\limits_{i=1}^3 \sin \beta_i \cdot steps_i \cdot r_i 
    \end{equation}

    \label{eqn:reverse-translation}
\end{subequations}
\begin{description}
    \item[$\Delta_x$] Déplacement du robot selon l'axe X, dans son prore référentiel.
    \item[$\Delta_y$] Déplacement du robot selon l'axe Y, dans son prore référentiel.
    \item[$steps$] Nombre de pas pour un tour de roue.
    \item[$steps_i$] Avancement de la roue $i$ en steps codeur.
    \item[$\beta_i$] Angle relatif au robot de l'orientation de la roue $i$.
    \item[$r_i$] Rayon de la roue $i$.
\end{description}


\subsubsection{Exemple en MATLAB}
\begin{lstlisting}
mov_x = cos(beta1)*steps1*r1/steps_turn
      + cos(beta2)*steps2*r2/steps_turn
      + cos(beta3)*steps3*r3/steps_turn;
mov_y = sin(beta1)*steps1*r1/steps_turn
      + sin(beta2)*steps2*r2/steps_turn
      + sin(beta3)*steps3*r3/steps_turn;

mov_x = mov_x * 2/3;
mov_y = mov_y * 2/3;
\end{lstlisting}

\subsection{Conversion en coordonnées table}
Pour appliquer l'équation \eqref{eqn:reverse-translation}, il faut d'abord appliquer
la matrice de rotation du robot à $\Delta_x$ et $\Delta_y$:

\begin{subequations}
    \begin{equation}
        X = X + \cos \left( \theta_R - \frac{\pi}{2} \right) \cdot \Delta_x - \sin \left( \theta_R - \frac{\pi}{2} \right) \cdot \Delta_y
    \end{equation}
    \begin{equation} 
        Y = Y + \sin \left( \theta_R - \frac{\pi}{2} \right) \cdot \Delta_x + \cos \left( \theta_R - \frac{\pi}{2} \right) \cdot \Delta_y
    \end{equation} 
    \label{eqn:coordinate-transform}
\end{subequations}

\subsubsection{Exemple en MATLAB}
\begin{lstlisting}
pos_x = pos_x + cos(heading-pi/2)*mov_x - ...
        sin(heading-pi/2)*mov_y;
pos_y = pos_y + sin(heading-pi/2)*mov_x + ...
        cos(heading-pi/2)*mov_y;
\end{lstlisting}


\end{document}
